\documentclass[a4paper]{scrartcl}

\usepackage[english]{babel}
\usepackage[utf8]{inputenc}
\usepackage{amsmath}
\usepackage{graphicx}
\usepackage[colorinlistoftodos]{todonotes}

\title{MPRI Web Data Management Project}
\subtitle{Aligning personal data : ccTV tracker using geolocalisation}

\author{Andreea Beica, Robin Champenois}

\date{\today}

\begin{document}
\maketitle

\section{Overview}
The purpose of the project is to locate surveillance cameras on whose footage a person could appear during their daily trajectories, using geolocalisation information gathered by generally available API's, such as Google Maps Location, or Open Street Map.

\section{Organisation}

The project is organised as follows:

\begin{itemize}
\item geolocalisation information collection: we use the Google Location History feature of Google Maps (accessible via a Google account) in order to track our trajectory and then retrieve and store all the visited locations in a suitable format (i.e., latitute + longitude)
\item surveillance camera database construction: sing the Open Street Map API, we construct a database of all the positions of surveillance cameras; we envisaged using more than one database, one for every region we would decide upon, in order anticipate large-scale storage or more efficient querying, but ultimately found that doing so did not improve things in a significant manner
\item external information collection: we then proceed to locate other potential surveillance cameras that might have caught footage of us but aren't contained in the information provided by Open Street Map; for this, we use the geolocalisation information gathered earlier on and the Google Places API to find places of interest we might have passed; we then filter these places, keeping only the ones that are usually equipped with ccTV cameras (e.g., ATM's, banks, public institutions, public transport stations)
\item locating potential footage: the final step is combining the personal data collected with the ccTV data in order to locate ccTV cameras that might have filmed us during our trajectories
\end{itemize}
\end{document}